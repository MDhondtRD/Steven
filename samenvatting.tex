Deze scriptie onderzoekt het gebruik van het framework React Native voor het ontwikkelen van native mobiele applicaties. Sinds de komst van React is er een grote mentaliteitswijziging op gebied van webontwikkeling. Standaarden werden heruitgevonden en niet door iedereen op evenveel applaus onthaald. React heeft in zijn korte bestaan zijn waarde aan front-end ontwikkeling bewezen en wint op het internet aan populariteit. In deze scriptie behandel ik React dan ook uitvoerig en maak ik een korte vergelijking met populaire frameworks als AngularJS en Backbone. React werd namelijk ontwikkeld om het rendementsprobleem van webpagina's met veel, veranderende data op een elegante wijze weer te geven. Dit blijkt ook uit de resultaten van de vergelijking. React werd door Facebook ontwikkeld omdat er vooral voor de eigen applicaties veel nood was aan een nieuwe manier om views op te bouwen. In deze scriptie komt u gedetailleerd te weten hoe een applicatie in React wordt opgebouwd en wat React zo krachtig maakt. 

In navolging van de populariteit van React op gebied van webontwikkeling werd een framework opgebouwd om native mobiele applicaties te ontwikkelen door gebruik te maken van de kracht van React. Het framework werd React Native gedoopt en uitgebracht in maart 2015 voor iOS ontwikkeling. De ondersteuning voor Android kwam er in september 2015. Ondanks het feit dat applicaties in React Native geschreven worden in Javascript is het toch zo dat de applicaties niet als hybride worden beschouwd maar als echte native applicaties. In de scriptie onderzoek ik hoe dit in zijn werk gaat en bespreek ik enkele voorbeelden van componenten en API's in React Native. Facebook wil met React Native het ontwikkelingsproces voor native applicaties vereenvoudigen en maakt hiervoor gebruik van reeds bestaande webtechnieken. Zo zullen webontwikkelaars sneller vertrouwd raken met het ontwikkelen van native applicaties en wordt debuggen en testen eenvoudiger en volledig analoog aan wat een webontwikkelaar gewoon is. 

In de scriptie onderzoek ik ook hoe stabiel React Native is. Uit het opbouwen van een voorbeeld applicatie met React Native is gebleken dat nog niet alles volledig op punt staat en dat voor het ontwikkelen van Android applicaties nog heel wat moet gebeuren. Omdat React Native nog jong is, zijn er delen die nog niet volledig af zijn waardoor die vaak tot kleine irritaties leiden. Facebook is een grote firma die zich in de wereld van de webontwikkeling al verscheidene keren heeft laten gelden en die bewezen heeft kwalitatieve applicaties en tools af te leveren. 
Met React Native zal dat ook niet anders zijn. Samen met de community zal Facebook ervoor zorgen dat er in de toekomst nog veel rond React Native te doen zal zijn. 
Toch zouden softwarebedrijven moeten investeren in React Native ontwikkelingen, voor Android en iOS wordt immers veel code hergebruikt waardoor het ontwikkelingsproces aanzienlijk verkort.


 