\chapter{Conclusie}
Toen ik voor het eerst over het concept van React Native leerde, stond ik er heel sceptisch tegenover, het was Javascript ontwikkeling maar toch geen Hybride applicatie ontwikkeling. Ik kon mij hier moeilijk bij neerleggen, mijn ervaringen met hybride ontwikkeling zijn heel kort geweest en ik deed de applicaties hierin vaak af als minderwaardige applicaties die zich willen voordoen als een echte native applicatie. Ik ben steeds van het idee geweest dat een echte native applicatie telkens in de taal wordt geschreven die de aanbieders van een OS voorleggen en die zij ondersteunen. Groot was echter mijn verbazing toen ik voor het eerst met React Native een applicatie ging ontwikkelen, want het voelde echt native aan en was het ook.\\
	In mijn onderzoek ben ik uiteraard eerst met React voor webontwikkeling begonnen om de syntax onder de knie te krijgen. Ik heb nooit echt graag met Javascript gewerkt waardoor ik bevooroordeeld was ten opzichte van React. Toch was het een aangename ervaring en het maken van een webapplicatie ging enorm vlot. Omdat React een verkorte versie is van Javascript heb ik de syntax heel snel onder de knie gekregen. Het is natuurlijk ook zo dat als een ontwikkelaar in React codeert hij enkel rekening moet houden met de view, waardoor het vaak kinderspel lijkt om een applicatie te bouwen. React is een goede manier om de view van een webapplicatie op te bouwen, maar voor server-side ontwikkeling zal er steeds een back-end programmeur nodig zijn. \\
	In al mijn enthousiasme ben ik aan React Native begonnen, de stukken die werden onderzocht in hoofdstuk \prettyref{chapter:workrn} resulteren in een gebundelde applicatie voor zowel iOS als voor Android. In het begin lag mijn focus op iOS, door mijn ervaring met Android en vooral Java ontwikkeling was ik heel gerust dat die stukken sneller zouden verwerkt worden. Het ontwikkelen voor iOS was enorm aangenaam en verliep heel vlot, de integratie met Xcode is volledig uitgewerkt door Facebook en hier heb ik geen problemen ondervonden. Het debuggen in chrome en in Xcode was zeer duidelijk en ik verstond onmiddellijk waarom mijn applicatie niet wou verdergaan. De applicatie liep heel vlot en ik kon het verschil niet merken met een applicatie die ik in Objective-C of Swift zou geschreven hebben. Toch is de manier om stijl te geven aan componenten niet echt een verbetering. Een webontwikkelaar is natuurlijk vertrouwd met CSS en zal dit waarschijnlijk sneller onder de knie krijgen als een mobiele applicatie ontwikkelaar. Ik miste toch een visuele tool om schermen op te bouwen zoals in Xcode voor iOS ontwikkeling of in Android studios voor Android. Ook is het zo dat de standaard schermopbouw voor IPhone of IPad applicaties, zoals aangeleverd door Apple, in native iOS ontwikkeling wordt voorzien voor de ontwikkelaar. Bij React Native is dit een moeilijker gegeven, je zou als ontwikkelaar de schermpbouw volledig zelf met stijlcomponenten moeten nabouwen om de standaard na te bootsen. Anderzijds, heb je als ontwikkelaar hierdoor veel meer vrijheid. \\
	Nadat ik een werkende applicatie van 1 pagina had, heb ik geprobeerd om die ook op Android te laten draaien. Ik maakte gebruik van een Genymotion emulator om mijn applicatie te laten draaien, dit volledige proces liep niet zo heel vlot. Volgens de richtlijnen zou door middel van een aantal commando's de applicatie moeten starten op de emulator. Dit ging niet onmiddellijk en ik heb via Stackoverflow een aantal configuraties moeten uitvoeren die Facebook niet had vermeld. Na een aantal keer proberen, is dit dan toch gelukt. Net als bij iOS was ik ook hier overtuigd van het resultaat en had ik dezelfde bemerkingen op gebied van stijlcomponenten. De ondersteuning van Android voor React Native is nog heel jong en dat merk je als ontwikkelaar. Er zijn heel wat componenten en API's die nog niet geïntegreerd werden in React Native en je moet goed zoeken om een antwoord te vinden op bepaalde problemen. Uiteraard wordt hier door Facebook en de community nog volop aan gewerkt.\\ 
	Toch zou ik alle bedrijven die zich met mobiele applicatie ontwikkeling bezighouden React Native aanbevelen. Er zijn inderdaad voorlopig nog een aantal mindere punten, maar de voordelen van React Native geven toch de doorslag. React is een Javascript library waarvan de syntax en de concepten heel snel aangeleerd zijn, zo kan iemand die Javascript kan coderen met een minimum aan inspanning onmiddellijk beginnen met React. Wanneer een ontwikkelaar React onder de knie heeft, is de stap naar React Native een kleine moeite, want veel van de concepten en werkwijzen worden hierin overgenomen. Een native applicatie ontwikkelaar zal dus met een kleine inspanning om React aan te leren, direct applicaties kunnen bouwen voor iOS en Android. Anderzijds geldt voor een webontwikkelaar dat wanneer hij vertrouwd geraakt met concepten van native applicatie ontwikkeling hij ook productief kan worden ingezet in native ontwikkeling, dankzij React Native. De opleiding van bekwame mensen is dus geen grote financiële kost voor een bedrijf maar een investering die snel rendeert.\\
	Daarnaast zijn er ook de voordelen om het ontwikkelingsproces te versnellen, zoals code herbruikbaarheid op zowel Android als iOS maar ook de mogelijkheid van onmiddellijk herladen van de applicatie zorgt voor tijdswinst. Doordat een React Native applicatie in Javascript geschreven wordt, zal een update van een applicatie die al in de applicatie stores aanwezig is onmiddellijk doorgevoerd worden. Dit zal de tevredenheid van de klant van de applicatie verhogen.
Er is nog steeds geen ondersteuning voor Windows Phone en er is voorlopig ook geen enkele indicatie dat dit ook in de nabije toekomst zal gebeuren. Het is ook een nadeel maar omdat het grootste deel van smartphone gebruikers met Android of iOS werkt, weegt dit niet heel zwaar door. Ik heb er ook alle vertrouwen in dat Facebook de ondersteuning voor Windows Phone alsnog zal integreren in React Native.


